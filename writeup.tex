\documentclass[12pt,letterpaper]{article}
\usepackage{verbatim}
\title{Homework 4 Write-Up}
\author{Aaron Okano, Jason Wong, Meenal Tambe, and Gowtham Vijayaragavan}

\begin{document}
\maketitle
\setcounter{page}{1}
\textit{\textbf{Write a COGENT explanation, along the lines of the ``Gowtham
and Noelle'' story in class, of how the threads take turns in this
application. Explain why it may be useful to have more threads than CPU cores
for this program but not for the prime finder above.}}
\newline

\linespread{1.5}
In the program, we needed to create a web probe that would read through
a list of URLs in a file using the number of threads that the user specifies.
The file---in our case, urls.txt---contains a list of URLs to probe. The program
takes three arguments, to be given in the following format:
\begin{verbatim}
% webprobe <URL file> <window width> <number of threads>
\end{verbatim}
The output of the program consists of the mean and standard deviation of the
{\bf window width} most recent probe times and the number of accesses performed
by each thread.

The {\bf main()} function begins by initializing all the variables and arrays
in accordance with the arguments. Then the ``probe'' threads, which each work to 
probe a URL, are created as well as the ``reporter'' thread. Afterwards, it simply
waits for the threads to finish---a condition which is not met, since the threads
immediately begin an infinite loop.

When each ``probe'' is created, it begins execution of the {\bf probe()} function. 
Because several threads will be accessing and writing to the same sets of data
(namely, the file pointer and the list of most recent accesses) {\bf probe()} makes
use of mutual exclusion objects (mutexes) which to coordinate the threads. The
mutexes themselves act as locks, which, after locked by the first thread to
encounter {\bf pthread\_mutex\_lock()}, prevent other threads from executing the
code following that point until the mutex is unlocked. The actual probing and
recording of access times is performed by first recording the time of day, forking
a \verb wget  process, recording the time of day after \verb wget  exits, and
finally, finding the difference between the two times and inserting the value
into an array of the {\bf window width} most recent times using the function
{\bf load\_recent()}. The function then continues to loop this cycle until
termination.

The single ``reporter'' thread, as can probably be guessed, executes the
{\bf reporter()} function. This function is set to sleep for 10.0 / number of
threads seconds using the usleep command. After sleeping, it first locks the
mutex to prevent the probe threads from accessing the data it needs to read.
Then it prints the mean and standard deviation of the most recent accesses
along with the number of accesses made by each probe thread. As with the
probes, {\bf reporter()} continues to loop until the program's termination.

Compared with the Primes program, webprobe benefits from having more threads
than available cores. The reason for this is that while each thread of the
Primes program runs computations the entire time, much of the time a probe
is executing, it is only waiting for \verb wget  to complete execution. Rather
than waste all that time waiting, the CPU can be put to use launching and timing
more instances of \verb wget .

\end{document}
